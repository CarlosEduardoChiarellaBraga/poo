% ******************************************************************
% Área do discente: Insira os seus dados e do problema selecionado.
% ******************************************************************
\def\discente{Carlos Eduardo Chiarella Braga}
\def\matricula{202200499}
\def\myprob{{03}} % Informe o número do problema selecionado.
% ******************************************************************

% ----------------------------------- Definições preliminares
\documentclass[12pt]{article}
%
\usepackage[brazil]{babel}
\usepackage[T1]{fontenc}
\usepackage[utf8]{inputenc}
\usepackage[a4paper,top=0.5cm,bottom=1.5cm,left=1.5cm,right=1.5cm,nohead,nofoot]{geometry}
%
\usepackage{xcolor}
\usepackage{amssymb}
\usepackage{mathtools}
\usepackage{enumitem}
\usepackage{booktabs}
\usepackage[breakable]{tcolorbox}
%
% ======================================= Início do documento
\begin{document}
% ------------------------------------------------- Cabeçalho
 \begin{tcolorbox}[rounded corners, colback=blue!3, colframe=blue!40!black]
  \footnotesize\textbf{Universidade Federal de Goiás -- UFG}\hfill \textsc{Pesquisa Operacional -- 2024/2}\\
  \footnotesize\textbf{Instituto de Informática -- INF\hfill Prof. Humberto J. Longo} -- \scriptsize\texttt{longo@ufg.br}
 \end{tcolorbox}\bigskip
%
% ------------------------------------------------- Atividade
\begin{tcolorbox}[rounded corners, colback=blue!3, colframe=blue!40!black, title=\textbf{Atividade AA-01 : \discente\ (\matricula)}]
 Nesta tarefa deve ser proposto um modelo de Programação Linear para um problema selecionado e apresentada uma solução ótima para o mesmo. Cada aluna(o) deve escolher, na descrição da atividade AA--01 na disciplina INF0313A\_BCC da plataforma Turing, um dos problemas ali listados. As especificações detalhadas dos problemas estão disponíveis no arquivo ``Modelagem de problemas como PL e PLI'' da Seção ``Listas de exercícios''.
\end{tcolorbox}\bigskip

%
% ------------------------------------- Descrição do problema selecionado.
%=========================================================================
\begin{tcolorbox}[rounded corners, colback=yellow!5, colframe=red!40!black, title=\textbf{Descrição do problema P-\myprob}]
%- - - - - - - - - - - - - - - - - - - - - - - - - - - - - - -
Uma fábrica de brinquedos educativos de madeira desenvolveu um boneco articulado composto por dezesseis peças independentes. Cada uma das peças pode ser pintada de uma das cores: verde, amarelo,
branco e azul. Além disso, há três “acessórios” opcionais (chapéu, óculos e saia) que também podem
ser pintados por qualquer uma dessas cores. A capacidade máxima de produção mensal é de 1000
bonecos, qualquer que seja a cor em que sejam produzidos e qualquer que seja o conjunto de acessórios
incorporados. No entanto, a produção de bonecos com acessórios exige um operário especializado que,
se produzir apenas bonecos com saia, consegue produzir independentemente da cor até 400 bonecos
por mês. Por outro lado, se produzir apenas bonecos com chapéu e óculos consegue produzir, independentemente da cor, até 500 bonecos por mês. A tabela a seguir representa, para cada um dos tipos de
bonecos, o lucro por unidade vendida e também a quantidade mínima a produzir por mês. Escreva um
modelo de P.L. para o planejamento da produção.  \begin{center}\small
   \begin{tabular}{lccc}
%    \toprule
    \cmidrule{4-4}
    \midrule
    Brinquedo & Lucro (R\$) & Qtde Mínima \\
    \midrule
     Boneco simples, branco               & 5       & 200 \\
     Boneco simples, colorido             & 4       & 100 \\
     Boneco com chapéu e óculos, branco   & 5       & 100 \\
     Boneco com chapéu e óculos, colorido & 15      & 100 \\
     Boneco com saia, branco              & 7       & 100 \\
     Boneco com saia, colorido            & 20      & 100 \\
    \bottomrule
   \end{tabular}
  \end{center}
\end{tcolorbox}\bigskip

%
% --------------------------------- Modelo PL para o problema selecionado.
%=========================================================================
\begin{tcolorbox}[rounded corners, breakable,colback=yellow!5, colframe=red!40!black, title=\textbf{Modelo PL para o problema P-\myprob}]
%- - - - - - - - - - - - - - - - - - - - - - - - - - - - - - -
 \begin{itemize}[topsep=0pt]
  \item Variáveis de decisão:
   \begin{description}[topsep=0pt,itemsep=0pt]
    \item [$x_1$\,:] qtde. a ser produzida de Boneco simples, branco;
    \item [$x_2$\,:] qtde. a ser produzida de Boneco simples, colorido;
    \item [$x_3$\,:] qtde. a ser produzida de Boneco com chapéu e óculos, branco.
    \item [$x_4$\,:] qtde. a ser produzida de Boneco com chapéu e óculos, colorido;
    \item [$x_5$\,:] qtde. a ser produzida de Boneco com saia, branco; e
    \item [$x_6$\,:] qtde. a ser produzida de Boneco com saia, colorido.
   \end{description}
%
  \item Objetivo (maximizar o lucro $z$ com a venda dos brinquedos produzidos):
   $$\max z= 5x_1 + 4x_2 + 5x_3 + 15x_4 + 7x_5 + 20x_6.$$
%
  \item Restrições:
   \begin{enumerate}[topsep=0pt,itemsep=0pt]
    \item Qtde. mínima de Boneco simples, branco: $x_1 \geqslant 200$.
    \item Qtde. mínima de Boneco simples, colorido: $x_2 \geqslant 100$.
    \item Qtde. mínima de Boneco com chapéu e óculos, branco: $x_3 \geqslant 100$.
    \item Qtde. mínima de Boneco com chapéu e óculos, colorido: $x_4 \geqslant 100$.
    \item Qtde. mínima de Boneco com saia, branco: $x_5 \geqslant 100$.
    \item Qtde. mínima de Boneco com saia, colorido: $x_6 \geqslant 100$.
    \item Limite de horas de trabalho: $(x_5 + x_6)/400 + (x_3+x_4)/500 \leqslant 1$.
    \item Capacidade máxima de brinquedos produzidos: $ x_1 + x_2 + x_3 + x_4 + x_5 + x_6 \leqslant 1000$.
   \end{enumerate}
%
    \item Modelo de programação linear:
    \small{
     $$\begin{array}{rcrcrcl}
        \multicolumn{7}{l}{\max\ z= 5x_1 + 4x_2 + 5x_3 + 15x_4 + 7x_5 + 20x_6.,} \\[4pt]
        \multicolumn{7}{l}{\text{sujeito a}} \\[4pt]
          x_1 & \geqslant & 200;\\
          x_2 & \geqslant & 100;\\
          x_3 & \geqslant & 100;\\
          x_4 & \geqslant & 100;\\
          x_5 & \geqslant & 100;\\
          x_6 & \geqslant & 100;\\
        x_1&+x_2 &+ x_3 &+ x_4 &+ x_5 &+ x_6 \leqslant & 1000;\\
          (x_5+x_6)/400& + & (x_3+x_4)/500 & \leqslant & 1.\\
       \end{array}$$
    }
%
    \item Solução ótima:
     $$
      \begin{array}{cccc}
      \toprule
      z^* (R\$) & x_1 & x_2 & x_3\\
      \midrule
      4.215,00  &  48 &   0 &  33\\
      \bottomrule
      \end{array}
     $$
 \end{itemize}
%- - - - - - - - - - - - - - - - - - - - - - - - - - - - - - -
\end{tcolorbox}
%=========================================================================
\end{document}
%
